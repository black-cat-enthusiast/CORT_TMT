% Options for packages loaded elsewhere
\PassOptionsToPackage{unicode}{hyperref}
\PassOptionsToPackage{hyphens}{url}
%
\documentclass[
]{book}
\usepackage{amsmath,amssymb}
\usepackage{iftex}
\ifPDFTeX
  \usepackage[T1]{fontenc}
  \usepackage[utf8]{inputenc}
  \usepackage{textcomp} % provide euro and other symbols
\else % if luatex or xetex
  \usepackage{unicode-math} % this also loads fontspec
  \defaultfontfeatures{Scale=MatchLowercase}
  \defaultfontfeatures[\rmfamily]{Ligatures=TeX,Scale=1}
\fi
\usepackage{lmodern}
\ifPDFTeX\else
  % xetex/luatex font selection
\fi
% Use upquote if available, for straight quotes in verbatim environments
\IfFileExists{upquote.sty}{\usepackage{upquote}}{}
\IfFileExists{microtype.sty}{% use microtype if available
  \usepackage[]{microtype}
  \UseMicrotypeSet[protrusion]{basicmath} % disable protrusion for tt fonts
}{}
\makeatletter
\@ifundefined{KOMAClassName}{% if non-KOMA class
  \IfFileExists{parskip.sty}{%
    \usepackage{parskip}
  }{% else
    \setlength{\parindent}{0pt}
    \setlength{\parskip}{6pt plus 2pt minus 1pt}}
}{% if KOMA class
  \KOMAoptions{parskip=half}}
\makeatother
\usepackage{xcolor}
\usepackage{color}
\usepackage{fancyvrb}
\newcommand{\VerbBar}{|}
\newcommand{\VERB}{\Verb[commandchars=\\\{\}]}
\DefineVerbatimEnvironment{Highlighting}{Verbatim}{commandchars=\\\{\}}
% Add ',fontsize=\small' for more characters per line
\usepackage{framed}
\definecolor{shadecolor}{RGB}{248,248,248}
\newenvironment{Shaded}{\begin{snugshade}}{\end{snugshade}}
\newcommand{\AlertTok}[1]{\textcolor[rgb]{0.94,0.16,0.16}{#1}}
\newcommand{\AnnotationTok}[1]{\textcolor[rgb]{0.56,0.35,0.01}{\textbf{\textit{#1}}}}
\newcommand{\AttributeTok}[1]{\textcolor[rgb]{0.13,0.29,0.53}{#1}}
\newcommand{\BaseNTok}[1]{\textcolor[rgb]{0.00,0.00,0.81}{#1}}
\newcommand{\BuiltInTok}[1]{#1}
\newcommand{\CharTok}[1]{\textcolor[rgb]{0.31,0.60,0.02}{#1}}
\newcommand{\CommentTok}[1]{\textcolor[rgb]{0.56,0.35,0.01}{\textit{#1}}}
\newcommand{\CommentVarTok}[1]{\textcolor[rgb]{0.56,0.35,0.01}{\textbf{\textit{#1}}}}
\newcommand{\ConstantTok}[1]{\textcolor[rgb]{0.56,0.35,0.01}{#1}}
\newcommand{\ControlFlowTok}[1]{\textcolor[rgb]{0.13,0.29,0.53}{\textbf{#1}}}
\newcommand{\DataTypeTok}[1]{\textcolor[rgb]{0.13,0.29,0.53}{#1}}
\newcommand{\DecValTok}[1]{\textcolor[rgb]{0.00,0.00,0.81}{#1}}
\newcommand{\DocumentationTok}[1]{\textcolor[rgb]{0.56,0.35,0.01}{\textbf{\textit{#1}}}}
\newcommand{\ErrorTok}[1]{\textcolor[rgb]{0.64,0.00,0.00}{\textbf{#1}}}
\newcommand{\ExtensionTok}[1]{#1}
\newcommand{\FloatTok}[1]{\textcolor[rgb]{0.00,0.00,0.81}{#1}}
\newcommand{\FunctionTok}[1]{\textcolor[rgb]{0.13,0.29,0.53}{\textbf{#1}}}
\newcommand{\ImportTok}[1]{#1}
\newcommand{\InformationTok}[1]{\textcolor[rgb]{0.56,0.35,0.01}{\textbf{\textit{#1}}}}
\newcommand{\KeywordTok}[1]{\textcolor[rgb]{0.13,0.29,0.53}{\textbf{#1}}}
\newcommand{\NormalTok}[1]{#1}
\newcommand{\OperatorTok}[1]{\textcolor[rgb]{0.81,0.36,0.00}{\textbf{#1}}}
\newcommand{\OtherTok}[1]{\textcolor[rgb]{0.56,0.35,0.01}{#1}}
\newcommand{\PreprocessorTok}[1]{\textcolor[rgb]{0.56,0.35,0.01}{\textit{#1}}}
\newcommand{\RegionMarkerTok}[1]{#1}
\newcommand{\SpecialCharTok}[1]{\textcolor[rgb]{0.81,0.36,0.00}{\textbf{#1}}}
\newcommand{\SpecialStringTok}[1]{\textcolor[rgb]{0.31,0.60,0.02}{#1}}
\newcommand{\StringTok}[1]{\textcolor[rgb]{0.31,0.60,0.02}{#1}}
\newcommand{\VariableTok}[1]{\textcolor[rgb]{0.00,0.00,0.00}{#1}}
\newcommand{\VerbatimStringTok}[1]{\textcolor[rgb]{0.31,0.60,0.02}{#1}}
\newcommand{\WarningTok}[1]{\textcolor[rgb]{0.56,0.35,0.01}{\textbf{\textit{#1}}}}
\usepackage{longtable,booktabs,array}
\usepackage{calc} % for calculating minipage widths
% Correct order of tables after \paragraph or \subparagraph
\usepackage{etoolbox}
\makeatletter
\patchcmd\longtable{\par}{\if@noskipsec\mbox{}\fi\par}{}{}
\makeatother
% Allow footnotes in longtable head/foot
\IfFileExists{footnotehyper.sty}{\usepackage{footnotehyper}}{\usepackage{footnote}}
\makesavenoteenv{longtable}
\usepackage{graphicx}
\makeatletter
\def\maxwidth{\ifdim\Gin@nat@width>\linewidth\linewidth\else\Gin@nat@width\fi}
\def\maxheight{\ifdim\Gin@nat@height>\textheight\textheight\else\Gin@nat@height\fi}
\makeatother
% Scale images if necessary, so that they will not overflow the page
% margins by default, and it is still possible to overwrite the defaults
% using explicit options in \includegraphics[width, height, ...]{}
\setkeys{Gin}{width=\maxwidth,height=\maxheight,keepaspectratio}
% Set default figure placement to htbp
\makeatletter
\def\fps@figure{htbp}
\makeatother
\setlength{\emergencystretch}{3em} % prevent overfull lines
\providecommand{\tightlist}{%
  \setlength{\itemsep}{0pt}\setlength{\parskip}{0pt}}
\setcounter{secnumdepth}{5}
\usepackage{booktabs}
\ifLuaTeX
  \usepackage{selnolig}  % disable illegal ligatures
\fi
\usepackage[]{natbib}
\bibliographystyle{plainnat}
\usepackage{bookmark}
\IfFileExists{xurl.sty}{\usepackage{xurl}}{} % add URL line breaks if available
\urlstyle{same}
\hypersetup{
  pdftitle={Corticosterone regulates the balance between freezing and rearing in defensive responses to predator threat},
  pdfauthor={Jennet Baumbach},
  hidelinks,
  pdfcreator={LaTeX via pandoc}}

\title{Corticosterone regulates the balance between freezing and rearing in defensive responses to predator threat}
\author{Jennet Baumbach}
\date{2025-12-09}

\begin{document}
\maketitle

{
\setcounter{tocdepth}{1}
\tableofcontents
}
\{-\{ \# About

This repository contains raw data and reproducible statistical analyses that are reported in the paper \emph{Corticosterone regulates the balance between freezing and rearing in defensive responses to predator threat} (2025).

\textbf{Authors}: Baumbach, J.L., Mui, C.Y.Y., Lionetti, A.M., and Martin, L.J.

\chapter*{Figure 1: TMT-Induced Freezing \& CPA}\label{figure-1-tmt-induced-freezing-cpa}
\addcontentsline{toc}{chapter}{Figure 1: TMT-Induced Freezing \& CPA}

\includegraphics[width=41.67in]{Figs/1_TMT_Frz}

\textbf{Figure 1}. (A) Mice are injected with saline or 50mg/kg metyrapone and exposed to 35μL of 10\% TMT 30 min later. (B \& C) Raster plots showing individual episodes of freezing and rearing during TMT exposure. (D) Metyrapone administration prevents freezing behavior and increases rearing during TMT exposure. (E) Saline-treated mice exhibit increased freezing during the five-minute exposure to TMT, which is prevented by metyrapone administration. (F) Metyrapone reduces the number of freezing episodes and increases the number of rearing bouts during TMT exposure. (G) The average length of freezing episodes is reduced by metyrapone. (H) Saline-treated mice show longer bouts of freezing during the five minutes of TMT exposure, an effect prevented by metyrapone administration. (I) Schematic of the single-exposure conditioned place aversion paradigm. (J) During the pre-test, mice show strong preferences for one side of the conditioning apparatus. (K) The single pairing with TMT abolishes basal preferences. (L) Both saline- and metyrapone-treated mice exhibit a reduction in CPA score from pre-test to post-test, and there is no group difference in the magnitude of the change. Data presented as mean value \(\pm\) SEM. * p \textless{} 0.05, ** p \textless{} 0.01, *** p \textless{} 0.001.

\section*{Behavioral Response During TMT Presentation}\label{behavioral-response-during-tmt-presentation}
\addcontentsline{toc}{section}{Behavioral Response During TMT Presentation}

\subsection*{Overall Time Spent Freezing / Rearing}\label{overall-time-spent-freezing-rearing}
\addcontentsline{toc}{subsection}{Overall Time Spent Freezing / Rearing}

To test whether CORT influenced behavioral responses to TMT, we pharmacologically inhibited its syntheses with metyrapone (Fig 1A).

\begin{Shaded}
\begin{Highlighting}[]
\NormalTok{b }\OtherTok{\textless{}{-}}\NormalTok{ a }\SpecialCharTok{\%\textgreater{}\%}
  \FunctionTok{group\_by}\NormalTok{(ID,Drug,Behavior) }\SpecialCharTok{\%\textgreater{}\%}
  \FunctionTok{summarise}\NormalTok{(}
    \AttributeTok{sum=}\FunctionTok{sum}\NormalTok{(Duration),}
    \AttributeTok{Number=}\FunctionTok{n}\NormalTok{(),}
\NormalTok{  ) }\SpecialCharTok{\%\textgreater{}\%}
  \FunctionTok{mutate}\NormalTok{(}\AttributeTok{Perc =}\NormalTok{ (sum }\SpecialCharTok{/} \DecValTok{300}\NormalTok{)}\SpecialCharTok{*}\DecValTok{100}\NormalTok{) }\SpecialCharTok{\%\textgreater{}\%}
  \FunctionTok{ungroup}\NormalTok{()}

\FunctionTok{anova\_test}\NormalTok{(}\AttributeTok{data =}\NormalTok{ b, }\AttributeTok{dv =}\NormalTok{ Perc, }\AttributeTok{within =}\NormalTok{ Behavior, }\AttributeTok{wid =}\NormalTok{ ID, }\AttributeTok{between =}\NormalTok{ Drug)}
\end{Highlighting}
\end{Shaded}

\begin{verbatim}
## ANOVA Table (type III tests)
## 
##          Effect DFn DFd      F            p p<.05   ges
## 1          Drug   1  20 17.422 0.0004680000     * 0.257
## 2      Behavior   1  20  1.684 0.2090000000       0.048
## 3 Drug:Behavior   1  20 67.193 0.0000000799     * 0.670
\end{verbatim}

CORT-synthesis inhibition resulted in a change in the TMT-evoked behavioral profile (drug X behavior interaction: F\textsubscript{1,20} = 67.19, p \textless{} 0.001). Compared to saline-injected controls, metyrapone-treated mice spent significantly less time freezing and more time rearing (Fig 1B,C,D).

\subsection*{Changes in Freezing Across the 5-minute Session}\label{changes-in-freezing-across-the-5-minute-session}
\addcontentsline{toc}{subsection}{Changes in Freezing Across the 5-minute Session}

\begin{Shaded}
\begin{Highlighting}[]
\NormalTok{b }\OtherTok{\textless{}{-}}\NormalTok{ data }\SpecialCharTok{\%\textgreater{}\%}
  \FunctionTok{na.omit}\NormalTok{() }\SpecialCharTok{\%\textgreater{}\%}
  \FunctionTok{filter}\NormalTok{(Behavior }\SpecialCharTok{!=} \StringTok{"groom"}\NormalTok{) }\SpecialCharTok{\%\textgreater{}\%}
  \FunctionTok{mutate}\NormalTok{(}\AttributeTok{Bins =} \FunctionTok{cut}\NormalTok{(}
\NormalTok{    Start\_clean,}
    \AttributeTok{breaks =} \FunctionTok{c}\NormalTok{(}\SpecialCharTok{{-}}\ConstantTok{Inf}\NormalTok{, }\DecValTok{60}\NormalTok{, }\DecValTok{120}\NormalTok{, }\DecValTok{180}\NormalTok{, }\DecValTok{240}\NormalTok{, }\SpecialCharTok{+}\ConstantTok{Inf}\NormalTok{),}
    \AttributeTok{labels =} \FunctionTok{c}\NormalTok{(}\StringTok{"1"}\NormalTok{, }\StringTok{"2"}\NormalTok{, }\StringTok{"3"}\NormalTok{, }\StringTok{"4"}\NormalTok{, }\StringTok{"5"}\NormalTok{)}
\NormalTok{  )) }\SpecialCharTok{\%\textgreater{}\%}
  \FunctionTok{group\_by}\NormalTok{(ID, Behavior, Drug, Bins) }\SpecialCharTok{\%\textgreater{}\%}
  \FunctionTok{summarise}\NormalTok{(}
    \AttributeTok{sum=}\FunctionTok{sum}\NormalTok{(Duration)}
\NormalTok{  ) }\SpecialCharTok{\%\textgreater{}\%} \FunctionTok{mutate}\NormalTok{(}\AttributeTok{Perc =}\NormalTok{ (sum }\SpecialCharTok{/} \DecValTok{60}\NormalTok{) }\SpecialCharTok{*} \DecValTok{100}\NormalTok{)}

\NormalTok{c }\OtherTok{\textless{}{-}}\NormalTok{ b }\SpecialCharTok{\%\textgreater{}\%}
  \FunctionTok{ungroup}\NormalTok{() }\SpecialCharTok{\%\textgreater{}\%}
  \FunctionTok{filter}\NormalTok{(Behavior }\SpecialCharTok{==} \StringTok{"freeze"}\NormalTok{)}

\FunctionTok{anova\_test}\NormalTok{(}\AttributeTok{data =}\NormalTok{ c, }\AttributeTok{dv =}\NormalTok{ Perc, }\AttributeTok{within =}\NormalTok{ Bins, }\AttributeTok{wid =}\NormalTok{ ID, }\AttributeTok{between =}\NormalTok{ Drug)}
\end{Highlighting}
\end{Shaded}

\begin{verbatim}
## ANOVA Table (type III tests)
## 
## $ANOVA
##      Effect DFn DFd      F        p p<.05   ges
## 1      Drug   1   8 35.432 0.000341     * 0.417
## 2      Bins   4  32  6.006 0.001000     * 0.386
## 3 Drug:Bins   4  32  4.800 0.004000     * 0.335
## 
## $`Mauchly's Test for Sphericity`
##      Effect     W     p p<.05
## 1      Bins 0.085 0.079      
## 2 Drug:Bins 0.085 0.079      
## 
## $`Sphericity Corrections`
##      Effect   GGe      DF[GG] p[GG] p[GG]<.05   HFe      DF[HF] p[HF] p[HF]<.05
## 1      Bins 0.519 2.08, 16.62 0.010         * 0.705 2.82, 22.56 0.004         *
## 2 Drug:Bins 0.519 2.08, 16.62 0.022         * 0.705 2.82, 22.56 0.011         *
\end{verbatim}

Freezing increased over the five minute TMT exposure in saline-injected mice, whereas metyrapone-treated mice did not show this time-dependent increase (drug X time interaction: F\textsubscript{4,32} = 4.80, p = 0.004; Fig 1E).

\subsection*{Number of Freezing / Rearing Episodes}\label{number-of-freezing-rearing-episodes}
\addcontentsline{toc}{subsection}{Number of Freezing / Rearing Episodes}

\begin{Shaded}
\begin{Highlighting}[]
\NormalTok{b }\OtherTok{\textless{}{-}}\NormalTok{ data }\SpecialCharTok{\%\textgreater{}\%}
  \FunctionTok{na.omit}\NormalTok{() }\SpecialCharTok{\%\textgreater{}\%}
  \FunctionTok{filter}\NormalTok{(Behavior }\SpecialCharTok{!=} \StringTok{"groom"}\NormalTok{) }\SpecialCharTok{\%\textgreater{}\%}
  \FunctionTok{group\_by}\NormalTok{(ID,Drug,Behavior) }\SpecialCharTok{\%\textgreater{}\%}
  \FunctionTok{summarise}\NormalTok{(}
    \AttributeTok{n=}\FunctionTok{n}\NormalTok{()}
\NormalTok{  ) }\SpecialCharTok{\%\textgreater{}\%}
  \FunctionTok{ungroup}\NormalTok{()}

\FunctionTok{anova\_test}\NormalTok{(}\AttributeTok{data =}\NormalTok{ b, }\AttributeTok{dv =}\NormalTok{ n, }\AttributeTok{within =}\NormalTok{ Behavior, }\AttributeTok{wid =}\NormalTok{ ID, }\AttributeTok{between =}\NormalTok{ Drug)}
\end{Highlighting}
\end{Shaded}

\begin{verbatim}
## ANOVA Table (type III tests)
## 
##          Effect DFn DFd      F             p p<.05   ges
## 1          Drug   1  20  1.071 0.31300000000       0.020
## 2      Behavior   1  20  1.618 0.21800000000       0.048
## 3 Drug:Behavior   1  20 97.941 0.00000000377     * 0.753
\end{verbatim}

\begin{Shaded}
\begin{Highlighting}[]
\NormalTok{b }\SpecialCharTok{\%\textgreater{}\%} 
  \FunctionTok{group\_by}\NormalTok{(Drug) }\SpecialCharTok{\%\textgreater{}\%}
  \FunctionTok{pairwise\_t\_test}\NormalTok{(n }\SpecialCharTok{\textasciitilde{}}\NormalTok{ Behavior, }\AttributeTok{paired =}\NormalTok{ T)}
\end{Highlighting}
\end{Shaded}

\begin{verbatim}
## # A tibble: 2 x 11
##   Drug       .y.   group1 group2    n1    n2 statistic    df          p    p.adj
## * <fct>      <chr> <chr>  <chr>  <int> <int>     <dbl> <dbl>      <dbl>    <dbl>
## 1 Saline     n     freeze rear      10    10      6.26     9 0.000147    1.47e-4
## 2 Metyrapone n     freeze rear      12    12     -7.87    11 0.00000762  7.62e-6
## # i 1 more variable: p.adj.signif <chr>
\end{verbatim}

\begin{Shaded}
\begin{Highlighting}[]
\NormalTok{b }\SpecialCharTok{\%\textgreater{}\%} 
  \FunctionTok{group\_by}\NormalTok{(Behavior) }\SpecialCharTok{\%\textgreater{}\%}
  \FunctionTok{pairwise\_t\_test}\NormalTok{(n }\SpecialCharTok{\textasciitilde{}}\NormalTok{ Drug)}
\end{Highlighting}
\end{Shaded}

\begin{verbatim}
## # A tibble: 2 x 10
##   Behavior .y.   group1 group2    n1    n2       p p.signif   p.adj p.adj.signif
## * <chr>    <chr> <chr>  <chr>  <int> <int>   <dbl> <chr>      <dbl> <chr>       
## 1 freeze   n     Saline Metyr~    10    12 6.29e-8 ****     6.29e-8 ****        
## 2 rear     n     Saline Metyr~    10    12 4.9 e-7 ****     4.9 e-7 ****
\end{verbatim}

In addition, metyrapone reduced freezing frequency and increased rearing frequency during TMT (drug X behavior interaction: F\textsubscript{1,20} = 97.94, p \textless{} 0.001; Fig 1F).

\subsection*{Bout Duration for Freezing and Rearing}\label{bout-duration-for-freezing-and-rearing}
\addcontentsline{toc}{subsection}{Bout Duration for Freezing and Rearing}

\begin{Shaded}
\begin{Highlighting}[]
\NormalTok{anova }\OtherTok{\textless{}{-}} \FunctionTok{aov}\NormalTok{(Duration }\SpecialCharTok{\textasciitilde{}}\NormalTok{ Behavior }\SpecialCharTok{*}\NormalTok{ Drug, }\AttributeTok{data =}\NormalTok{ a)}
\FunctionTok{summary}\NormalTok{(anova)}
\end{Highlighting}
\end{Shaded}

\begin{verbatim}
##                 Df Sum Sq Mean Sq F value       Pr(>F)    
## Behavior         1     58   57.66   15.71 0.0000792220 ***
## Drug             1     82   81.82   22.29 0.0000026787 ***
## Behavior:Drug    1    114  113.94   31.04 0.0000000324 ***
## Residuals     1015   3726    3.67                         
## ---
## Signif. codes:  0 '***' 0.001 '**' 0.01 '*' 0.05 '.' 0.1 ' ' 1
\end{verbatim}

\begin{Shaded}
\begin{Highlighting}[]
\NormalTok{a }\SpecialCharTok{\%\textgreater{}\%} 
  \FunctionTok{group\_by}\NormalTok{(Behavior) }\SpecialCharTok{\%\textgreater{}\%}
  \FunctionTok{pairwise\_t\_test}\NormalTok{(Duration }\SpecialCharTok{\textasciitilde{}}\NormalTok{ Drug)}
\end{Highlighting}
\end{Shaded}

\begin{verbatim}
## # A tibble: 2 x 10
##   Behavior .y.   group1 group2    n1    n2       p p.signif   p.adj p.adj.signif
## * <chr>    <chr> <chr>  <chr>  <int> <int>   <dbl> <chr>      <dbl> <chr>       
## 1 freeze   Dura~ Saline Metyr~   344   109 4.13e-7 ****     4.13e-7 ****        
## 2 rear     Dura~ Saline Metyr~   140   426 7.07e-1 ns       7.07e-1 ns
\end{verbatim}

The average length of freezing epochs was longer for saline- compared to metyrapone-treated mice (p \textless{} 0.001; Fig 1G), but there was no difference in the average length of rearing episodes (p = 0.71; Fig 1G).

\subsection*{Changes in Bout Length of Freezing Across the 5-minute Session}\label{changes-in-bout-length-of-freezing-across-the-5-minute-session}
\addcontentsline{toc}{subsection}{Changes in Bout Length of Freezing Across the 5-minute Session}

\begin{Shaded}
\begin{Highlighting}[]
\NormalTok{b }\OtherTok{\textless{}{-}}\NormalTok{ a }\SpecialCharTok{\%\textgreater{}\%}
  \FunctionTok{na.omit}\NormalTok{() }\SpecialCharTok{\%\textgreater{}\%}
  \FunctionTok{mutate}\NormalTok{(}\AttributeTok{Bins =} \FunctionTok{cut}\NormalTok{(}
\NormalTok{    Start\_clean,}
    \AttributeTok{breaks =} \DecValTok{5}\NormalTok{,}
    \AttributeTok{labels=}\FunctionTok{c}\NormalTok{(}\StringTok{"1"}\NormalTok{,}\StringTok{"2"}\NormalTok{,}\StringTok{"3"}\NormalTok{,}\StringTok{"4"}\NormalTok{,}\StringTok{"5"}\NormalTok{)}
\NormalTok{  )) }\SpecialCharTok{\%\textgreater{}\%}
  \FunctionTok{group\_by}\NormalTok{(ID, Behavior, Drug, Bins) }\SpecialCharTok{\%\textgreater{}\%}
  \FunctionTok{filter}\NormalTok{(Behavior }\SpecialCharTok{==} \StringTok{"freeze"}\NormalTok{)}

\NormalTok{anova }\OtherTok{\textless{}{-}} \FunctionTok{aov}\NormalTok{(Duration}\SpecialCharTok{\textasciitilde{}}\NormalTok{Bins}\SpecialCharTok{*}\NormalTok{Drug,}\AttributeTok{data=}\NormalTok{b)}
\FunctionTok{summary}\NormalTok{(anova)}
\end{Highlighting}
\end{Shaded}

\begin{verbatim}
##              Df Sum Sq Mean Sq F value             Pr(>F)    
## Bins          4  451.9  112.97  18.071 0.0000000000000925 ***
## Drug          1  203.2  203.17  32.499 0.0000000218006710 ***
## Bins:Drug     4  113.2   28.30   4.527            0.00136 ** 
## Residuals   443 2769.5    6.25                               
## ---
## Signif. codes:  0 '***' 0.001 '**' 0.01 '*' 0.05 '.' 0.1 ' ' 1
\end{verbatim}

The length of freezing episodes increased across the five minute TMT exposure for saline-injected mice, but not for metyrapone-treated mice (drug X time interaction: F\textsubscript{4,443} = 4.53, p = 0.001; Fig 1H). These results indicate that blocking CORT synthesis during TMT exposure shifts the defensive response from predominantly freezing to primarily rearing.

\section*{Conditioned Place Aversion to TMT}\label{conditioned-place-aversion-to-tmt}
\addcontentsline{toc}{section}{Conditioned Place Aversion to TMT}

Next, to understand whether mice developed an aversion to the context where they previously encountered TMT, we conducted a conditioned place aversion test (Fig 1I).

\subsection*{Baseline Preferences}\label{baseline-preferences}
\addcontentsline{toc}{subsection}{Baseline Preferences}

\begin{Shaded}
\begin{Highlighting}[]
\FunctionTok{t.test}\NormalTok{(BL\_pref\_data}\SpecialCharTok{$}\NormalTok{Tm\_White, BL\_pref\_data}\SpecialCharTok{$}\NormalTok{Tm\_Black, }\AttributeTok{paired =}\NormalTok{ T)}
\end{Highlighting}
\end{Shaded}

\begin{verbatim}
## 
##  Paired t-test
## 
## data:  BL_pref_data$Tm_White and BL_pref_data$Tm_Black
## t = -5.049, df = 23, p-value = 0.00004131
## alternative hypothesis: true mean difference is not equal to 0
## 95 percent confidence interval:
##  -46.77726 -19.58690
## sample estimates:
## mean difference 
##       -33.18208
\end{verbatim}

Mice exhibited robust side preferences at baseline (t\textsubscript{23} = 5.05, p \textless{} 0.001), and TMT was paired with the side of the apparatus that each mouse exhibited a baseline preference for (Fig 1J).

\subsection*{Test Session}\label{test-session}
\addcontentsline{toc}{subsection}{Test Session}

\subsubsection*{CPA for Each Drug Group}\label{cpa-for-each-drug-group}
\addcontentsline{toc}{subsubsection}{CPA for Each Drug Group}

\begin{Shaded}
\begin{Highlighting}[]
\NormalTok{s }\OtherTok{\textless{}{-}}\NormalTok{ Test\_data[Test\_data}\SpecialCharTok{$}\NormalTok{Drug }\SpecialCharTok{==} \StringTok{"Saline"}\NormalTok{, ]}
\NormalTok{m }\OtherTok{\textless{}{-}}\NormalTok{ Test\_data[Test\_data}\SpecialCharTok{$}\NormalTok{Drug }\SpecialCharTok{==} \StringTok{"Metyrapone"}\NormalTok{, ]}

\FunctionTok{t.test}\NormalTok{(s}\SpecialCharTok{$}\NormalTok{Pref\_tm, s}\SpecialCharTok{$}\NormalTok{Non\_Pref\_tm, }\AttributeTok{paired =}\NormalTok{ T)}
\end{Highlighting}
\end{Shaded}

\begin{verbatim}
## 
##  Paired t-test
## 
## data:  s$Pref_tm and s$Non_Pref_tm
## t = -1.2068, df = 11, p-value = 0.2528
## alternative hypothesis: true mean difference is not equal to 0
## 95 percent confidence interval:
##  -60.60959  17.68126
## sample estimates:
## mean difference 
##       -21.46417
\end{verbatim}

\begin{Shaded}
\begin{Highlighting}[]
\FunctionTok{t.test}\NormalTok{(m}\SpecialCharTok{$}\NormalTok{Pref\_tm, m}\SpecialCharTok{$}\NormalTok{Non\_Pref\_tm, }\AttributeTok{paired =}\NormalTok{ T)}
\end{Highlighting}
\end{Shaded}

\begin{verbatim}
## 
##  Paired t-test
## 
## data:  m$Pref_tm and m$Non_Pref_tm
## t = -1.3417, df = 11, p-value = 0.2067
## alternative hypothesis: true mean difference is not equal to 0
## 95 percent confidence interval:
##  -53.31831  12.93164
## sample estimates:
## mean difference 
##       -20.19333
\end{verbatim}

Preferences for the initially preferred side of the apparatus were abolished by the single pairing with TMT for both saline- and metyrapone-treated mice (both p values \textgreater{} 0.2 during the post-test; Fig 1K).

\subsubsection*{Change in Preference Between the Baseline and Test Sessions}\label{change-in-preference-between-the-baseline-and-test-sessions}
\addcontentsline{toc}{subsubsection}{Change in Preference Between the Baseline and Test Sessions}

\begin{Shaded}
\begin{Highlighting}[]
\FunctionTok{t.test}\NormalTok{(a}\SpecialCharTok{$}\NormalTok{BL\_CPA\_score, a}\SpecialCharTok{$}\NormalTok{Post\_CPA\_score, }\AttributeTok{paired =}\NormalTok{ T)}
\end{Highlighting}
\end{Shaded}

\begin{verbatim}
## 
##  Paired t-test
## 
## data:  a$BL_CPA_score and a$Post_CPA_score
## t = 2.6027, df = 10, p-value = 0.02637
## alternative hypothesis: true mean difference is not equal to 0
## 95 percent confidence interval:
##  0.09237031 1.19126470
## sample estimates:
## mean difference 
##       0.6418175
\end{verbatim}

\begin{Shaded}
\begin{Highlighting}[]
\FunctionTok{t.test}\NormalTok{(b}\SpecialCharTok{$}\NormalTok{BL\_CPA\_score, b}\SpecialCharTok{$}\NormalTok{Post\_CPA\_score, }\AttributeTok{paired =}\NormalTok{ T)}
\end{Highlighting}
\end{Shaded}

\begin{verbatim}
## 
##  Paired t-test
## 
## data:  b$BL_CPA_score and b$Post_CPA_score
## t = 4.0866, df = 11, p-value = 0.0018
## alternative hypothesis: true mean difference is not equal to 0
## 95 percent confidence interval:
##  0.2529397 0.8434275
## sample estimates:
## mean difference 
##       0.5481836
\end{verbatim}

Both groups exhibited a reduction in CPA score after the single TMT exposure, indicating conditioned place aversion (saline t\textsubscript{10} = 2.60, p = 0.026; metyrapone t\textsubscript{11} = 4.09, p = 0.002; Fig 1L).

\subsubsection*{Compare CPA Scores for Saline and Metyrapone Mice}\label{compare-cpa-scores-for-saline-and-metyrapone-mice}
\addcontentsline{toc}{subsubsection}{Compare CPA Scores for Saline and Metyrapone Mice}

\begin{Shaded}
\begin{Highlighting}[]
\FunctionTok{t.test}\NormalTok{(delta\_CPA\_score }\SpecialCharTok{\textasciitilde{}}\NormalTok{ Drug, }\AttributeTok{data =}\NormalTok{ A\_A)}
\end{Highlighting}
\end{Shaded}

\begin{verbatim}
## 
##  Welch Two Sample t-test
## 
## data:  delta_CPA_score by Drug
## t = -0.33355, df = 15.556, p-value = 0.7432
## alternative hypothesis: true difference in means between group Saline and group Metyrapone is not equal to 0
## 95 percent confidence interval:
##  -0.6901155  0.5028477
## sample estimates:
##     mean in group Saline mean in group Metyrapone 
##               -0.6418175               -0.5481836
\end{verbatim}

Metyrapone treatment during TMT did not alter the magnitude of the change in CPA score between sessions (p = 0.74; Fig 1L).

\chapter*{Figure 2: Basal Freezing Before \& After TMT}\label{figure-2-basal-freezing-before-after-tmt}
\addcontentsline{toc}{chapter}{Figure 2: Basal Freezing Before \& After TMT}

\includegraphics[width=41.67in]{Figs/2_Basal_before_after}

\textbf{Figure 2}. (A) Mice are tested for basal levels of freezing and rearing before and 10 days after a single five-minute exposure to TMT. (B) Raster plots showing basal freezing before and 10 days after the single exposure to TMT. (C) Mice exhibit an increase in basal freezing after TMT exposure. (D) Metyrapone treatment during the single exposure to TMT reduces freezing in the absence of TMT 10 days later. (E) TMT increases basal freezing frequency, but (F) metyrapone treatment during TMT reduces freezing frequency 10 days later. (G) Mice treated with saline during TMT exhibit an increase in the average duration of freezing episodes, whereas metyrapone-treated mice do not. Data presented as mean value \(\pm\)
SEM. ** p \textless{} 0.01, *** p \textless{} 0.001.

\section*{Time Spent Freezing Before / After TMT}\label{time-spent-freezing-before-after-tmt}
\addcontentsline{toc}{section}{Time Spent Freezing Before / After TMT}

To examine whether inhibiting CORT synthesis during TMT exposure had lasting effects on defensive behavior, we measured freezing and rearing 10 days after the TMT exposure in the absence of any drug treatment or aversive stimulus (recall test).

\begin{Shaded}
\begin{Highlighting}[]
\NormalTok{a }\OtherTok{\textless{}{-}}\NormalTok{ data }\SpecialCharTok{\%\textgreater{}\%}
  \FunctionTok{na.omit}\NormalTok{() }\SpecialCharTok{\%\textgreater{}\%}
  \FunctionTok{filter}\NormalTok{(Behavior }\SpecialCharTok{==} \StringTok{"freeze"}\NormalTok{) }\SpecialCharTok{\%\textgreater{}\%}
  \FunctionTok{group\_by}\NormalTok{(ID, Drug, Behavior, session) }\SpecialCharTok{\%\textgreater{}\%}
  \FunctionTok{summarise}\NormalTok{(}
    \AttributeTok{sum=}\FunctionTok{sum}\NormalTok{(Duration),}
    \AttributeTok{freq =} \FunctionTok{n}\NormalTok{(),}
\NormalTok{  ) }\SpecialCharTok{\%\textgreater{}\%} \FunctionTok{mutate}\NormalTok{(}\AttributeTok{Perc =}\NormalTok{ sum }\SpecialCharTok{/} \DecValTok{300} \SpecialCharTok{*} \DecValTok{100}\NormalTok{) }

\NormalTok{d }\OtherTok{\textless{}{-}}\NormalTok{ a[a}\SpecialCharTok{$}\NormalTok{Behavior }\SpecialCharTok{==} \StringTok{"freeze"}\NormalTok{, ] }\SpecialCharTok{\%\textgreater{}\%}
  \FunctionTok{ungroup}\NormalTok{()}

\NormalTok{res }\OtherTok{\textless{}{-}} \FunctionTok{anova\_test}\NormalTok{(}\AttributeTok{data =}\NormalTok{ d, }\AttributeTok{dv =}\NormalTok{ Perc, }\AttributeTok{between =}\NormalTok{ Drug, }\AttributeTok{within =}\NormalTok{ session, }\AttributeTok{wid =}\NormalTok{ ID)}
\FunctionTok{get\_anova\_table}\NormalTok{(res)}
\end{Highlighting}
\end{Shaded}

\begin{verbatim}
## ANOVA Table (type II tests)
## 
##         Effect DFn DFd      F           p p<.05   ges
## 1         Drug   1  22  4.648 0.042000000     * 0.113
## 2      session   1  22 58.894 0.000000117     * 0.515
## 3 Drug:session   1  22 11.586 0.003000000     * 0.173
\end{verbatim}

\begin{Shaded}
\begin{Highlighting}[]
\NormalTok{d }\SpecialCharTok{\%\textgreater{}\%}
  \FunctionTok{group\_by}\NormalTok{(Drug) }\SpecialCharTok{\%\textgreater{}\%}
  \FunctionTok{pairwise\_t\_test}\NormalTok{(Perc }\SpecialCharTok{\textasciitilde{}}\NormalTok{ session) }
\end{Highlighting}
\end{Shaded}

\begin{verbatim}
## # A tibble: 2 x 10
##   Drug     .y.   group1 group2    n1    n2       p p.signif   p.adj p.adj.signif
## * <fct>    <chr> <chr>  <chr>  <int> <int>   <dbl> <chr>      <dbl> <chr>       
## 1 Saline   Perc  10 da~ Basel~    12    12 1.10e-5 ****     1.10e-5 ****        
## 2 Metyrap~ Perc  10 da~ Basel~    12    12 7.89e-4 ***      7.89e-4 ***
\end{verbatim}

Both saline- and metyrapone-treated mice displayed an increase in freezing time when the baseline session and the recall session were compared (main effect of test session: F1,22 = 58.89, p \textless{} 0.001; Fig 2A,B,C).

\begin{Shaded}
\begin{Highlighting}[]
\NormalTok{d }\SpecialCharTok{\%\textgreater{}\%}
  \FunctionTok{group\_by}\NormalTok{(session) }\SpecialCharTok{\%\textgreater{}\%}
  \FunctionTok{pairwise\_t\_test}\NormalTok{(Perc }\SpecialCharTok{\textasciitilde{}}\NormalTok{ Drug)}
\end{Highlighting}
\end{Shaded}

\begin{verbatim}
## # A tibble: 2 x 10
##   session    .y.   group1 group2    n1    n2      p p.signif  p.adj p.adj.signif
## * <chr>      <chr> <chr>  <chr>  <int> <int>  <dbl> <chr>     <dbl> <chr>       
## 1 10 days a~ Perc  Saline Metyr~    12    12 0.0075 **       0.0075 **          
## 2 Baseline   Perc  Saline Metyr~    12    12 0.415  ns       0.415  ns
\end{verbatim}

However, metyrapone administration during TMT reduced the amount of time spent freezing during the recall test relative to saline-injected controls (drug X session interaction: F1,22 = 11.57, p = 0.003; effect of drug on freezing during the recall test: p = 0.008; Fig 2D).

\begin{Shaded}
\begin{Highlighting}[]
\NormalTok{res }\OtherTok{\textless{}{-}} \FunctionTok{anova\_test}\NormalTok{(}\AttributeTok{data =}\NormalTok{ d, }\AttributeTok{dv =}\NormalTok{ freq, }\AttributeTok{between =}\NormalTok{ Drug, }\AttributeTok{within =}\NormalTok{ session, }\AttributeTok{wid =}\NormalTok{ ID)}
\FunctionTok{get\_anova\_table}\NormalTok{(res)}
\end{Highlighting}
\end{Shaded}

\begin{verbatim}
## ANOVA Table (type II tests)
## 
##         Effect DFn DFd      F             p p<.05   ges
## 1         Drug   1  22  7.025 0.01500000000     * 0.145
## 2      session   1  22 94.985 0.00000000192     * 0.669
## 3 Drug:session   1  22  9.645 0.00500000000     * 0.171
\end{verbatim}

\section*{Freezing Frequency}\label{freezing-frequency}
\addcontentsline{toc}{section}{Freezing Frequency}

Freezing frequency was increased by the single TMT exposure (main effect of session: F1,22 = 94.98, p \textless{} 0.001; Fig 2E).

\begin{Shaded}
\begin{Highlighting}[]
\NormalTok{d }\SpecialCharTok{\%\textgreater{}\%}
  \FunctionTok{group\_by}\NormalTok{(session) }\SpecialCharTok{\%\textgreater{}\%}
  \FunctionTok{pairwise\_t\_test}\NormalTok{(freq }\SpecialCharTok{\textasciitilde{}}\NormalTok{ Drug)}
\end{Highlighting}
\end{Shaded}

\begin{verbatim}
## # A tibble: 2 x 10
##   session  .y.   group1 group2    n1    n2       p p.signif   p.adj p.adj.signif
## * <chr>    <chr> <chr>  <chr>  <int> <int>   <dbl> <chr>      <dbl> <chr>       
## 1 10 days~ freq  Saline Metyr~    12    12 0.00111 **       0.00111 **          
## 2 Baseline freq  Saline Metyr~    12    12 0.831   ns       0.831   ns
\end{verbatim}

Mice treated with metyrapone during TMT froze less often than saline controls (drug X session interaction: F1,22 = 9.64, p = 0.005; effect for drug during the recall test: p = 0.001; Fig 2F).

\section*{Freezing Bout Length}\label{freezing-bout-length}
\addcontentsline{toc}{section}{Freezing Bout Length}

\begin{Shaded}
\begin{Highlighting}[]
\NormalTok{a }\OtherTok{\textless{}{-}}\NormalTok{ data[data}\SpecialCharTok{$}\NormalTok{Behavior }\SpecialCharTok{==} \StringTok{"freeze"}\NormalTok{, ]}

\NormalTok{res }\OtherTok{\textless{}{-}} \FunctionTok{aov}\NormalTok{(Duration }\SpecialCharTok{\textasciitilde{}}\NormalTok{ Drug }\SpecialCharTok{*}\NormalTok{ session, }\AttributeTok{data =}\NormalTok{ a)}
\FunctionTok{summary}\NormalTok{(res)}
\end{Highlighting}
\end{Shaded}

\begin{verbatim}
##                Df Sum Sq Mean Sq F value   Pr(>F)    
## Drug            1    2.3   2.266   6.511 0.010831 *  
## session         1    4.7   4.657  13.378 0.000264 ***
## Drug:session    1    4.6   4.579  13.153 0.000298 ***
## Residuals    1357  472.4   0.348                     
## ---
## Signif. codes:  0 '***' 0.001 '**' 0.01 '*' 0.05 '.' 0.1 ' ' 1
\end{verbatim}

\begin{Shaded}
\begin{Highlighting}[]
\NormalTok{a }\SpecialCharTok{\%\textgreater{}\%}
  \FunctionTok{group\_by}\NormalTok{(Drug) }\SpecialCharTok{\%\textgreater{}\%}
  \FunctionTok{pairwise\_t\_test}\NormalTok{(Duration }\SpecialCharTok{\textasciitilde{}}\NormalTok{ session) }\SpecialCharTok{\%\textgreater{}\%}
    \FunctionTok{mutate}\NormalTok{(}\AttributeTok{p =} \FunctionTok{formatC}\NormalTok{(p, }\AttributeTok{format =} \StringTok{"f"}\NormalTok{, }\AttributeTok{digits =} \DecValTok{6}\NormalTok{))}
\end{Highlighting}
\end{Shaded}

\begin{verbatim}
## # A tibble: 2 x 10
##   Drug       .y.   group1 group2    n1    n2 p     p.signif   p.adj p.adj.signif
##   <fct>      <chr> <chr>  <chr>  <int> <int> <chr> <chr>      <dbl> <chr>       
## 1 Saline     Dura~ 10 da~ Basel~   614   165 0.00~ ****     2.93e-7 ****        
## 2 Metyrapone Dura~ 10 da~ Basel~   407   175 0.95~ ns       9.55e-1 ns
\end{verbatim}

The average length of bouts of freezing increased post-TMT for saline-treated mice (drug X session interaction: F1,1357 = 13.15, p \textless{} 0.001; Effect of session for saline-treated mice: p \textless{} 0.001; Fig 2G), whereas there was no change in the length of freezing episodes before and after TMT for the metyrapone-injected group (p = 0.95; Fig 2G).

\chapter*{Figure 3: A 30-Minute TMT Exposure}\label{figure-3-a-30-minute-tmt-exposure}
\addcontentsline{toc}{chapter}{Figure 3: A 30-Minute TMT Exposure}

\includegraphics[width=41.67in]{Figs/3_30_min_TMT}

\textbf{Figure 3.} (A) Mice are injected with saline or metyrapone 30 min before a 30-min exposure to TMT. (B \& C) Raster plots showing individual episodes of freezing and rearing across a prolonged (30-min) TMT exposure. (D) Metyrapone treatment reduces freezing and increases rearing across the session. (E) Metyrapone treatment causes a slower onset of the freezing response and a reversal of the temporal pattern of rearing across the prolonged exposure. (F) Metyrapone does not alter freezing frequency during the prolonged session and increases rearing frequency. (G) Metyrapone reduces the average duration of freezing episodes. (H) Metyrapone treatment reduces the average length of freezing episodes during the first 25 min of the prolonged TMT session. Data presented as mean value \(\pm\) SEM. * p \textless{} 0.05, ** p \textless{} 0.01, *** p \textless{} 0.001.

\section*{Time Spent Freezing \& Rearing During Prolonged TMT Exposure}\label{time-spent-freezing-rearing-during-prolonged-tmt-exposure}
\addcontentsline{toc}{section}{Time Spent Freezing \& Rearing During Prolonged TMT Exposure}

To investigate whether CORT-synthesis inhibition prevented freezing in mice or altered temporal patterns of defensive responses, we conducted a prolonged (30-min) exposure to TMT.

\begin{Shaded}
\begin{Highlighting}[]
\NormalTok{a }\OtherTok{\textless{}{-}}\NormalTok{ data  }\SpecialCharTok{\%\textgreater{}\%}
  \FunctionTok{filter}\NormalTok{(Behavior }\SpecialCharTok{!=} \StringTok{"groom"}\NormalTok{) }\SpecialCharTok{\%\textgreater{}\%}
  \FunctionTok{filter}\NormalTok{(Behavior }\SpecialCharTok{!=} \StringTok{"start"}\NormalTok{) }\SpecialCharTok{\%\textgreater{}\%}
  \FunctionTok{filter}\NormalTok{(Behavior }\SpecialCharTok{!=} \StringTok{"stop"}\NormalTok{) }\SpecialCharTok{\%\textgreater{}\%}
  \FunctionTok{group\_by}\NormalTok{(ID,Drug,Behavior) }\SpecialCharTok{\%\textgreater{}\%}
  \FunctionTok{summarise}\NormalTok{(}
    \AttributeTok{sum=}\FunctionTok{sum}\NormalTok{(Duration),}
    \AttributeTok{Number=}\FunctionTok{n}\NormalTok{(),}
\NormalTok{  ) }\SpecialCharTok{\%\textgreater{}\%}
  \FunctionTok{mutate}\NormalTok{(}\AttributeTok{Perc =}\NormalTok{ (sum }\SpecialCharTok{/} \DecValTok{1800}\NormalTok{)}\SpecialCharTok{*}\DecValTok{100}\NormalTok{) }\SpecialCharTok{\%\textgreater{}\%}
  \FunctionTok{mutate}\NormalTok{(}\AttributeTok{Av\_DUR =}\NormalTok{ (sum }\SpecialCharTok{/}\NormalTok{ Number)) }

\NormalTok{res }\OtherTok{\textless{}{-}} \FunctionTok{aov}\NormalTok{(Perc }\SpecialCharTok{\textasciitilde{}}\NormalTok{ Drug }\SpecialCharTok{*}\NormalTok{ Behavior, }\AttributeTok{data =}\NormalTok{ a)}
\FunctionTok{summary}\NormalTok{(res)}
\end{Highlighting}
\end{Shaded}

\begin{verbatim}
##               Df Sum Sq Mean Sq F value     Pr(>F)    
## Drug           1    552     552   3.482    0.07253 .  
## Behavior       1   5235    5235  33.047 0.00000361 ***
## Drug:Behavior  1   1545    1545   9.750    0.00414 ** 
## Residuals     28   4436     158                       
## ---
## Signif. codes:  0 '***' 0.001 '**' 0.01 '*' 0.05 '.' 0.1 ' ' 1
\end{verbatim}

Metyrapone altered defensive behaviors during the 30-min session (drug X behavior interaction: F1,28\,=\,9.75, p\,=\,0.004; Fig. 3A).

\begin{Shaded}
\begin{Highlighting}[]
\NormalTok{a }\SpecialCharTok{\%\textgreater{}\%}
  \FunctionTok{group\_by}\NormalTok{(Behavior) }\SpecialCharTok{\%\textgreater{}\%}
  \FunctionTok{pairwise\_t\_test}\NormalTok{(Perc }\SpecialCharTok{\textasciitilde{}}\NormalTok{ Drug)}
\end{Highlighting}
\end{Shaded}

\begin{verbatim}
## # A tibble: 2 x 10
##   Behavior .y.   group1 group2    n1    n2       p p.signif   p.adj p.adj.signif
## * <chr>    <chr> <chr>  <chr>  <int> <int>   <dbl> <chr>      <dbl> <chr>       
## 1 freeze   Perc  Saline Metyr~     8     8 0.024   *        0.024   *           
## 2 rear     Perc  Saline Metyr~     8     8 0.00209 **       0.00209 **
\end{verbatim}

Specifically, metyrapone reduced time spent freezing (p\,=\,0.024) and increased time spent rearing (p\,=\,0.002; Fig. 3B,C,D). Metyrapone-treated mice still exhibited freezing, but it appeared much later in the session compared to saline-treated controls (Fig. 3E).

\section*{Freezing Frequency and Bout Duration}\label{freezing-frequency-and-bout-duration}
\addcontentsline{toc}{section}{Freezing Frequency and Bout Duration}

\begin{Shaded}
\begin{Highlighting}[]
\NormalTok{a }\OtherTok{\textless{}{-}}\NormalTok{ data }\SpecialCharTok{\%\textgreater{}\%}
  \FunctionTok{filter}\NormalTok{(Behavior }\SpecialCharTok{==} \StringTok{"freeze"}\NormalTok{) }\SpecialCharTok{\%\textgreater{}\%}
  \FunctionTok{group\_by}\NormalTok{(ID, Drug) }\SpecialCharTok{\%\textgreater{}\%}
  \FunctionTok{summarise}\NormalTok{(}\AttributeTok{n =} \FunctionTok{n}\NormalTok{())}

\FunctionTok{t.test}\NormalTok{(n }\SpecialCharTok{\textasciitilde{}}\NormalTok{ Drug, }\AttributeTok{data =}\NormalTok{ a, }\AttributeTok{var.equal =}\NormalTok{ T)}
\end{Highlighting}
\end{Shaded}

\begin{verbatim}
## 
##  Two Sample t-test
## 
## data:  n by Drug
## t = 1.6718, df = 14, p-value = 0.1168
## alternative hypothesis: true difference in means between group Saline and group Metyrapone is not equal to 0
## 95 percent confidence interval:
##  -5.940881 47.940881
## sample estimates:
##     mean in group Saline mean in group Metyrapone 
##                   94.125                   73.125
\end{verbatim}

\begin{Shaded}
\begin{Highlighting}[]
\NormalTok{b }\OtherTok{\textless{}{-}}\NormalTok{ data[data}\SpecialCharTok{$}\NormalTok{Behavior }\SpecialCharTok{==} \StringTok{"freeze"}\NormalTok{, ]}
\FunctionTok{t.test}\NormalTok{(}\AttributeTok{data =}\NormalTok{ b, Duration }\SpecialCharTok{\textasciitilde{}}\NormalTok{ Drug)}
\end{Highlighting}
\end{Shaded}

\begin{verbatim}
## 
##  Welch Two Sample t-test
## 
## data:  Duration by Drug
## t = 4.0768, df = 1335.9, p-value = 0.00004836
## alternative hypothesis: true difference in means between group Saline and group Metyrapone is not equal to 0
## 95 percent confidence interval:
##  1.602460 4.575125
## sample estimates:
##     mean in group Saline mean in group Metyrapone 
##                 8.272509                 5.183716
\end{verbatim}

Although freezing frequency did not differ between groups (p\,=\,0.11; Fig. 3F), the average length of freezing episodes was reduced by metyrapone (t\textsubscript{1335}\,=\,4.07, p\,\textless\,0.001; Fig. 3F).

\begin{Shaded}
\begin{Highlighting}[]
\NormalTok{a }\OtherTok{\textless{}{-}}\NormalTok{ data}
\NormalTok{b }\OtherTok{\textless{}{-}}\NormalTok{ a }\SpecialCharTok{\%\textgreater{}\%}
  \FunctionTok{na.omit}\NormalTok{() }\SpecialCharTok{\%\textgreater{}\%}
  \FunctionTok{filter}\NormalTok{(Behavior }\SpecialCharTok{!=} \StringTok{"groom"}\NormalTok{) }\SpecialCharTok{\%\textgreater{}\%}
  \FunctionTok{mutate}\NormalTok{(}\AttributeTok{Bins =} \FunctionTok{cut}\NormalTok{(}
\NormalTok{    Start\_clean,}
    \AttributeTok{breaks =} \DecValTok{6}\NormalTok{,}
    \AttributeTok{labels=}\FunctionTok{c}\NormalTok{(}\StringTok{"1"}\NormalTok{,}\StringTok{"2"}\NormalTok{,}\StringTok{"3"}\NormalTok{,}\StringTok{"4"}\NormalTok{,}\StringTok{"5"}\NormalTok{,}\StringTok{"6"}\NormalTok{)}
\NormalTok{  )) }\SpecialCharTok{\%\textgreater{}\%}
  \FunctionTok{group\_by}\NormalTok{(ID, Behavior, Drug, Bins) }

\NormalTok{res }\OtherTok{\textless{}{-}} \FunctionTok{aov}\NormalTok{(Duration }\SpecialCharTok{\textasciitilde{}}\NormalTok{ Drug }\SpecialCharTok{*}\NormalTok{ Bins, }\AttributeTok{data =}\NormalTok{ b)}
\FunctionTok{summary}\NormalTok{(res)}
\end{Highlighting}
\end{Shaded}

\begin{verbatim}
##               Df Sum Sq Mean Sq F value               Pr(>F)    
## Drug           1   5771    5771   51.22     0.00000000000110 ***
## Bins           5  11104    2221   19.71 < 0.0000000000000002 ***
## Drug:Bins      5   7377    1475   13.10     0.00000000000134 ***
## Residuals   2353 265100     113                                 
## ---
## Signif. codes:  0 '***' 0.001 '**' 0.01 '*' 0.05 '.' 0.1 ' ' 1
\end{verbatim}

\begin{Shaded}
\begin{Highlighting}[]
\NormalTok{b }\SpecialCharTok{\%\textgreater{}\%}
  \FunctionTok{group\_by}\NormalTok{(Bins) }\SpecialCharTok{\%\textgreater{}\%}
  \FunctionTok{pairwise\_t\_test}\NormalTok{(Duration }\SpecialCharTok{\textasciitilde{}}\NormalTok{ Drug)}
\end{Highlighting}
\end{Shaded}

\begin{verbatim}
## # A tibble: 6 x 10
##   Bins  .y.    group1 group2    n1    n2        p p.signif    p.adj p.adj.signif
## * <fct> <chr>  <chr>  <chr>  <int> <int>    <dbl> <chr>       <dbl> <chr>       
## 1 1     Durat~ Saline Metyr~   298   292 1.18e- 8 ****     1.18e- 8 ****        
## 2 2     Durat~ Saline Metyr~   187   282 2.89e-18 ****     2.89e-18 ****        
## 3 3     Durat~ Saline Metyr~   175   259 1.09e-15 ****     1.09e-15 ****        
## 4 4     Durat~ Saline Metyr~   114   196 2.42e- 5 ****     2.42e- 5 ****        
## 5 5     Durat~ Saline Metyr~   114   152 3.5 e- 1 ns       3.5 e- 1 ns          
## 6 6     Durat~ Saline Metyr~   159   137 2.26e- 2 *        2.26e- 2 *
\end{verbatim}

Both groups showed a gradual increase in freezing episode duration across the 30-min session, but saline-treated mice had longer freezing bouts than metyrapone-injected mice for the first 25\,min (all p\,\textless\,0.001; Fig. 3H).

\section*{Rearing Frequency and Bout Duration}\label{rearing-frequency-and-bout-duration}
\addcontentsline{toc}{section}{Rearing Frequency and Bout Duration}

\begin{Shaded}
\begin{Highlighting}[]
\NormalTok{a }\OtherTok{\textless{}{-}}\NormalTok{ data }\SpecialCharTok{\%\textgreater{}\%}
  \FunctionTok{filter}\NormalTok{(Behavior }\SpecialCharTok{==} \StringTok{"rear"}\NormalTok{) }\SpecialCharTok{\%\textgreater{}\%}
  \FunctionTok{group\_by}\NormalTok{(ID, Drug) }\SpecialCharTok{\%\textgreater{}\%}
  \FunctionTok{summarise}\NormalTok{(}\AttributeTok{n =} \FunctionTok{n}\NormalTok{())}

\FunctionTok{t.test}\NormalTok{(n }\SpecialCharTok{\textasciitilde{}}\NormalTok{ Drug, }\AttributeTok{data =}\NormalTok{ a, }\AttributeTok{var.equal =}\NormalTok{ T)}
\end{Highlighting}
\end{Shaded}

\begin{verbatim}
## 
##  Two Sample t-test
## 
## data:  n by Drug
## t = -3.937, df = 14, p-value = 0.001489
## alternative hypothesis: true difference in means between group Saline and group Metyrapone is not equal to 0
## 95 percent confidence interval:
##  -84.76955 -24.98045
## sample estimates:
##     mean in group Saline mean in group Metyrapone 
##                   36.750                   91.625
\end{verbatim}

\begin{Shaded}
\begin{Highlighting}[]
\NormalTok{b }\OtherTok{\textless{}{-}}\NormalTok{ data[data}\SpecialCharTok{$}\NormalTok{Behavior }\SpecialCharTok{==} \StringTok{"rear"}\NormalTok{, ]}
\FunctionTok{t.test}\NormalTok{(}\AttributeTok{data =}\NormalTok{ b, Duration }\SpecialCharTok{\textasciitilde{}}\NormalTok{ Drug)}
\end{Highlighting}
\end{Shaded}

\begin{verbatim}
## 
##  Welch Two Sample t-test
## 
## data:  Duration by Drug
## t = 0.11388, df = 568.98, p-value = 0.9094
## alternative hypothesis: true difference in means between group Saline and group Metyrapone is not equal to 0
## 95 percent confidence interval:
##  -0.1782223  0.2001608
## sample estimates:
##     mean in group Saline mean in group Metyrapone 
##                 1.852272                 1.841303
\end{verbatim}

Metyrapone-treated mice reared more frequently than saline treated controls (p\,=\,0.001; Fig. 3F), although there was no difference in the average length of rearing episodes (p\,=\,0.91; Fig. 3G,H).

\chapter*{Figure 4: A Second TMT Exposure}\label{figure-4-a-second-tmt-exposure}
\addcontentsline{toc}{chapter}{Figure 4: A Second TMT Exposure}

\includegraphics[width=41.67in]{Figs/4_30_min_mice_second_TMT}

\textbf{Figure 4}. (A) Mice are exposed to TMT for a second, five-minute test 10\,days after the 30-min exposure depicted in Fig. 3. (B \& C) Raster plots showing individual episodes of freezing and rearing during the second five-minute exposure to TMT. (D) Metyrapone during the first TMT exposure reduces freezing during the second encounter. (E) Freezing behavior increases across the five-minute session for both groups, but mice injected with saline before the first TMT exposure exhibit more freezing during every minute of the second exposure than mice treated with metyrapone during the first TMT session. (F) Mice treated with metyrapone during the first TMT session exhibit more rearing than those treated with saline. (G) The average duration of freezing episodes during the second TMT session is reduced for mice treated with metyrapone during the initial encounter with TMT. (H) The average length of freezing episodes increases across the five minutes for both groups, but mice treated with saline during the initial TMT encounter exhibit longer bouts of freezing than metyrapone-treated mice from the third minute onwards. Data presented as mean value \(\pm\) SEM. *** p\,\textless\,0.001.

\section*{Time Spent Rearing and Freezing}\label{time-spent-rearing-and-freezing}
\addcontentsline{toc}{section}{Time Spent Rearing and Freezing}

To determine whether CORT sensitizes defensive responses to repeated threats, we conducted a second five-minute exposure to TMT 10\,days after the initial session (Fig. 4A).

\begin{Shaded}
\begin{Highlighting}[]
\NormalTok{a }\OtherTok{\textless{}{-}}\NormalTok{ data  }\SpecialCharTok{\%\textgreater{}\%}
  \FunctionTok{filter}\NormalTok{(Behavior }\SpecialCharTok{==} \StringTok{"freeze"}\NormalTok{) }\SpecialCharTok{\%\textgreater{}\%}
  \FunctionTok{group\_by}\NormalTok{(ID,Drug,Behavior) }\SpecialCharTok{\%\textgreater{}\%}
  \FunctionTok{summarise}\NormalTok{(}
    \AttributeTok{sum=}\FunctionTok{sum}\NormalTok{(Duration),}
    \AttributeTok{Number=}\FunctionTok{n}\NormalTok{(),}
\NormalTok{  ) }\SpecialCharTok{\%\textgreater{}\%}
  \FunctionTok{mutate}\NormalTok{(}\AttributeTok{Perc =}\NormalTok{ (sum }\SpecialCharTok{/} \DecValTok{300}\NormalTok{)}\SpecialCharTok{*}\DecValTok{100}\NormalTok{) }\SpecialCharTok{\%\textgreater{}\%}
  \FunctionTok{mutate}\NormalTok{(}\AttributeTok{Av\_DUR =}\NormalTok{ (sum }\SpecialCharTok{/}\NormalTok{ Number)) }

\FunctionTok{t.test}\NormalTok{(Perc }\SpecialCharTok{\textasciitilde{}}\NormalTok{ Drug, }\AttributeTok{data =}\NormalTok{ a, }\AttributeTok{var.equal =}\NormalTok{ T)}
\end{Highlighting}
\end{Shaded}

\begin{verbatim}
## 
##  Two Sample t-test
## 
## data:  Perc by Drug
## t = 5.3862, df = 14, p-value = 0.00009597
## alternative hypothesis: true difference in means between group Saline and group Metyrapone is not equal to 0
## 95 percent confidence interval:
##  13.15398 30.56143
## sample estimates:
##     mean in group Saline mean in group Metyrapone 
##                 37.38829                 15.53058
\end{verbatim}

Mice that had received metyrapone during their first TMT exposure spent a lower percentage of time freezing during the second exposure than saline-treated controls (t14\,=\,5.39, p\,\textless\,0.001; Fig. 4B,C,D).

\section*{Changes In Freezing Across the Session}\label{changes-in-freezing-across-the-session}
\addcontentsline{toc}{section}{Changes In Freezing Across the Session}

\begin{Shaded}
\begin{Highlighting}[]
\NormalTok{b }\OtherTok{\textless{}{-}}\NormalTok{ data }\SpecialCharTok{\%\textgreater{}\%}
  \FunctionTok{na.omit}\NormalTok{() }\SpecialCharTok{\%\textgreater{}\%}
  \FunctionTok{filter}\NormalTok{(Behavior }\SpecialCharTok{==} \StringTok{"freeze"}\NormalTok{) }\SpecialCharTok{\%\textgreater{}\%}
  \FunctionTok{mutate}\NormalTok{(}\AttributeTok{Bins =} \FunctionTok{cut}\NormalTok{(}
\NormalTok{    Start\_clean,}
    \AttributeTok{breaks =} \FunctionTok{c}\NormalTok{(}\SpecialCharTok{{-}}\ConstantTok{Inf}\NormalTok{, }\DecValTok{60}\NormalTok{, }\DecValTok{120}\NormalTok{, }\DecValTok{180}\NormalTok{, }\DecValTok{240}\NormalTok{, }\SpecialCharTok{+}\ConstantTok{Inf}\NormalTok{),}
    \AttributeTok{labels =} \FunctionTok{c}\NormalTok{(}\StringTok{"1"}\NormalTok{, }\StringTok{"2"}\NormalTok{, }\StringTok{"3"}\NormalTok{, }\StringTok{"4"}\NormalTok{, }\StringTok{"5"}\NormalTok{)}
\NormalTok{  )) }\SpecialCharTok{\%\textgreater{}\%}
  \FunctionTok{group\_by}\NormalTok{(ID, Behavior, Drug, Bins) }\SpecialCharTok{\%\textgreater{}\%}
  \FunctionTok{summarise}\NormalTok{(}
    \AttributeTok{sum=}\FunctionTok{sum}\NormalTok{(Duration)}
\NormalTok{  ) }\SpecialCharTok{\%\textgreater{}\%} \FunctionTok{mutate}\NormalTok{(}\AttributeTok{Perc =}\NormalTok{ (sum }\SpecialCharTok{/} \DecValTok{60}\NormalTok{) }\SpecialCharTok{*} \DecValTok{100}\NormalTok{)}

\NormalTok{res }\OtherTok{\textless{}{-}} \FunctionTok{aov}\NormalTok{(Perc }\SpecialCharTok{\textasciitilde{}}\NormalTok{ Drug }\SpecialCharTok{*}\NormalTok{ Bins, }\AttributeTok{data =}\NormalTok{ b)}
\FunctionTok{summary}\NormalTok{(res)}
\end{Highlighting}
\end{Shaded}

\begin{verbatim}
##             Df Sum Sq Mean Sq F value          Pr(>F)    
## Drug         1   9094    9094  56.576 0.0000000001496 ***
## Bins         4  14060    3515  21.869 0.0000000000109 ***
## Drug:Bins    4   1283     321   1.995           0.105    
## Residuals   69  11091     161                            
## ---
## Signif. codes:  0 '***' 0.001 '**' 0.01 '*' 0.05 '.' 0.1 ' ' 1
\end{verbatim}

Although both groups progressively increased the amount of time spent freezing over the five-minute session, saline-treated mice consistently froze more than metyrapone-treated mice throughout (main effect of drug: F\textsubscript{1,69}\,=\,56.58, p\,\textless\,0.001; Fig. 4D).

\section*{Freezing Frequency}\label{freezing-frequency-1}
\addcontentsline{toc}{section}{Freezing Frequency}

\begin{Shaded}
\begin{Highlighting}[]
\NormalTok{b }\OtherTok{\textless{}{-}}\NormalTok{ data }\SpecialCharTok{\%\textgreater{}\%}
  \FunctionTok{na.omit}\NormalTok{() }\SpecialCharTok{\%\textgreater{}\%}
  \FunctionTok{filter}\NormalTok{(Behavior }\SpecialCharTok{==} \StringTok{"freeze"}\NormalTok{)}

\FunctionTok{t.test}\NormalTok{(Duration }\SpecialCharTok{\textasciitilde{}}\NormalTok{ Drug, }\AttributeTok{data =}\NormalTok{ b)}
\end{Highlighting}
\end{Shaded}

\begin{verbatim}
## 
##  Welch Two Sample t-test
## 
## data:  Duration by Drug
## t = 6.9453, df = 509.62, p-value = 0.00000000001156
## alternative hypothesis: true difference in means between group Saline and group Metyrapone is not equal to 0
## 95 percent confidence interval:
##  0.9096138 1.6272037
## sample estimates:
##     mean in group Saline mean in group Metyrapone 
##                 2.451691                 1.183283
\end{verbatim}

There was no difference in the frequency of freezing (Fig. 4F), but saline-treated mice exhibited longer freezing bouts than those treated with metyrapone during the first TMT exposure (t\textsubscript{510}\,=\,6.95, p\,\textless\,0.001; Fig. 4G).

\section*{Freezing Bout Duration}\label{freezing-bout-duration}
\addcontentsline{toc}{section}{Freezing Bout Duration}

\begin{Shaded}
\begin{Highlighting}[]
\NormalTok{b }\OtherTok{\textless{}{-}}\NormalTok{ data }\SpecialCharTok{\%\textgreater{}\%}
  \FunctionTok{na.omit}\NormalTok{() }\SpecialCharTok{\%\textgreater{}\%}
  \FunctionTok{mutate}\NormalTok{(}\AttributeTok{Bins =} \FunctionTok{cut}\NormalTok{(}
\NormalTok{    Start\_clean,}
    \AttributeTok{breaks =} \DecValTok{5}\NormalTok{,}
    \AttributeTok{labels=}\FunctionTok{c}\NormalTok{(}\StringTok{"1"}\NormalTok{,}\StringTok{"2"}\NormalTok{,}\StringTok{"3"}\NormalTok{,}\StringTok{"4"}\NormalTok{,}\StringTok{"5"}\NormalTok{)}
\NormalTok{  )) }\SpecialCharTok{\%\textgreater{}\%}
  \FunctionTok{group\_by}\NormalTok{(ID, Behavior, Drug, Bins) }\SpecialCharTok{\%\textgreater{}\%}
  \FunctionTok{filter}\NormalTok{(Behavior }\SpecialCharTok{==} \StringTok{"freeze"}\NormalTok{)}

\NormalTok{res }\OtherTok{\textless{}{-}} \FunctionTok{aov}\NormalTok{(Duration }\SpecialCharTok{\textasciitilde{}}\NormalTok{ Drug }\SpecialCharTok{*}\NormalTok{ Bins, }\AttributeTok{data =}\NormalTok{ b)}
\FunctionTok{summary}\NormalTok{(res)}
\end{Highlighting}
\end{Shaded}

\begin{verbatim}
##              Df Sum Sq Mean Sq F value               Pr(>F)    
## Drug          1    272  272.37   56.02    0.000000000000227 ***
## Bins          4    782  195.49   40.20 < 0.0000000000000002 ***
## Drug:Bins     4    209   52.13   10.72    0.000000019899505 ***
## Residuals   671   3263    4.86                                 
## ---
## Signif. codes:  0 '***' 0.001 '**' 0.01 '*' 0.05 '.' 0.1 ' ' 1
\end{verbatim}

Both groups progressively increased freezing episode durations across the session, but the magnitude of the increase was greater for saline-treated mice, which had an intact CORT response during the first TMT exposure (drug X time interaction: F\textsubscript{4,671}\,=\,10.72 p\,\textless\,0.001; Fig. 4H).

\chapter*{Figure 5: Restraint / Footshock Experiments}\label{figure-5-restraint-footshock-experiments}
\addcontentsline{toc}{chapter}{Figure 5: Restraint / Footshock Experiments}

\includegraphics[width=41.67in]{Figs/5_mimic_panel}

\textbf{Figure 5}. (A) Schematic depiction of physical restraint 30\,min before a single exposure to TMT. (B \& C) Raster plots showing individual episodes of freezing and rearing during the TMT session. (D) Physical restraint before TMT increases rearing and decreases freezing overall. (E) Freezing and rearing during TMT are broken down by minute. (F) Restraint reduces rearing frequency during TMT exposure. (G) Mice exposed to physical restraint before TMT exhibit longer freezing episodes, relative to non-stressed controls. (H) The length of freezing episodes was longer for mice subjected to physical restraint from minute two onward with TMT. (I) Schematic depiction of footshock protocol 10\,days before TMT exposure. (J \& K) Raster plots showing freezing and rearing during the TMT exposure, respectively. (L) Mice that had previously been subjected to footshock spend more time freezing and less time rearing than shock-naive mice. (M) Freezing and rearing during each minute of the TMT exposure. (N) Shock-primed mice freeze more often and rear less frequently than shock-naive controls. (O) On average, bouts of rearing are shorter for shock-primed mice than shock-naive . (P) Length of freezing episodes shown for each minute of the test. Data presented as mean value \(\pm\) SEM. * p\,\textless\,0.05, ** p\,\textless\,0.01, *** p\,\textless\,0.001.

Since CORT suppression during the first TMT exposure blunted the enhancement of defensive responses upon re-exposure (Fig. 4), we sought to determine whether this effect was specific to TMT sensitization or indicative of a broader stress-induced increase in threat sensitivity. To test this, we examined whether prior exposure to restraint stress or footshock (two distinct stressors) would similarly amplify defensive responses to TMT.

\section*{Restraint Priming}\label{restraint-priming}
\addcontentsline{toc}{section}{Restraint Priming}

\subsection*{Time Spent Rearing and Freezing During TMT}\label{time-spent-rearing-and-freezing-during-tmt}
\addcontentsline{toc}{subsection}{Time Spent Rearing and Freezing During TMT}

\begin{Shaded}
\begin{Highlighting}[]
\NormalTok{a }\OtherTok{\textless{}{-}}\NormalTok{ data  }\SpecialCharTok{\%\textgreater{}\%}
  \FunctionTok{filter}\NormalTok{(Behavior }\SpecialCharTok{\%in\%} \FunctionTok{c}\NormalTok{(}\StringTok{"rear"}\NormalTok{, }\StringTok{"freeze"}\NormalTok{)) }\SpecialCharTok{\%\textgreater{}\%}
  \FunctionTok{group\_by}\NormalTok{(ID,Condition,Behavior) }\SpecialCharTok{\%\textgreater{}\%}
  \FunctionTok{summarise}\NormalTok{(}
    \AttributeTok{sum=}\FunctionTok{sum}\NormalTok{(Duration),}
    \AttributeTok{Number=}\FunctionTok{n}\NormalTok{(),}
\NormalTok{  ) }\SpecialCharTok{\%\textgreater{}\%}
  \FunctionTok{mutate}\NormalTok{(}\AttributeTok{Perc =}\NormalTok{ (sum }\SpecialCharTok{/} \DecValTok{300}\NormalTok{)}\SpecialCharTok{*}\DecValTok{100}\NormalTok{) }\SpecialCharTok{\%\textgreater{}\%}
  \FunctionTok{mutate}\NormalTok{(}\AttributeTok{Av\_DUR =}\NormalTok{ (sum }\SpecialCharTok{/}\NormalTok{ Number)) }

\NormalTok{res }\OtherTok{\textless{}{-}} \FunctionTok{aov}\NormalTok{(Perc }\SpecialCharTok{\textasciitilde{}}\NormalTok{ Condition }\SpecialCharTok{*}\NormalTok{ Behavior, }\AttributeTok{data =}\NormalTok{ a)}
\FunctionTok{summary}\NormalTok{(res)}
\end{Highlighting}
\end{Shaded}

\begin{verbatim}
##                    Df Sum Sq Mean Sq F value        Pr(>F)    
## Condition           1    500     500   7.313        0.0112 *  
## Behavior            1   4569    4569  66.880 0.00000000396 ***
## Condition:Behavior  1   1866    1866  27.317 0.00001228407 ***
## Residuals          30   2049      68                          
## ---
## Signif. codes:  0 '***' 0.001 '**' 0.01 '*' 0.05 '.' 0.1 ' ' 1
\end{verbatim}

Mice subjected to restraint stress before TMT (Fig. 5A) exhibited more freezing and less rearing than control mice (drug X condition interaction: F1,30\,=\,27.32, p\,\textless\,0.001; Fig. 5B,C,D).

\subsection*{Time Spent Freezing Binned by Minute}\label{time-spent-freezing-binned-by-minute}
\addcontentsline{toc}{subsection}{Time Spent Freezing Binned by Minute}

\begin{Shaded}
\begin{Highlighting}[]
\NormalTok{a }\OtherTok{\textless{}{-}}\NormalTok{ data }\SpecialCharTok{\%\textgreater{}\%}
  \FunctionTok{na.omit}\NormalTok{() }\SpecialCharTok{\%\textgreater{}\%}
  \FunctionTok{mutate}\NormalTok{(}\AttributeTok{Bins =} \FunctionTok{cut}\NormalTok{(}
\NormalTok{    Start\_clean,}
    \AttributeTok{breaks =} \DecValTok{5}\NormalTok{,}
    \AttributeTok{labels=}\FunctionTok{c}\NormalTok{(}\StringTok{"1"}\NormalTok{,}\StringTok{"2"}\NormalTok{,}\StringTok{"3"}\NormalTok{,}\StringTok{"4"}\NormalTok{,}\StringTok{"5"}\NormalTok{)}
\NormalTok{  )) }\SpecialCharTok{\%\textgreater{}\%}
  \FunctionTok{group\_by}\NormalTok{(ID, Behavior,Condition, Bins) }\SpecialCharTok{\%\textgreater{}\%}
  \FunctionTok{summarise}\NormalTok{(}
    \AttributeTok{sum =} \FunctionTok{sum}\NormalTok{ (Duration)}
\NormalTok{  ) }\SpecialCharTok{\%\textgreater{}\%}
  \FunctionTok{mutate}\NormalTok{(}\AttributeTok{Perc =}\NormalTok{ (sum }\SpecialCharTok{/} \DecValTok{60}\NormalTok{)}\SpecialCharTok{*}\DecValTok{100}\NormalTok{ ) }\SpecialCharTok{\%\textgreater{}\%}
  \FunctionTok{filter}\NormalTok{(Behavior }\SpecialCharTok{!=} \StringTok{"groom"}\NormalTok{) }\SpecialCharTok{\%\textgreater{}\%}
  \FunctionTok{filter}\NormalTok{(Behavior }\SpecialCharTok{!=} \StringTok{"rear"}\NormalTok{)}

\NormalTok{res }\OtherTok{\textless{}{-}} \FunctionTok{aov}\NormalTok{(Perc }\SpecialCharTok{\textasciitilde{}}\NormalTok{ Condition }\SpecialCharTok{*}\NormalTok{ Bins, }\AttributeTok{data =}\NormalTok{ a)}
\FunctionTok{summary}\NormalTok{(res)}
\end{Highlighting}
\end{Shaded}

\begin{verbatim}
##                Df Sum Sq Mean Sq F value         Pr(>F)    
## Condition       1   9647    9647   32.97 0.000000200643 ***
## Bins            4  20618    5155   17.62 0.000000000369 ***
## Condition:Bins  4   3301     825    2.82          0.031 *  
## Residuals      73  21360     293                           
## ---
## Signif. codes:  0 '***' 0.001 '**' 0.01 '*' 0.05 '.' 0.1 ' ' 1
\end{verbatim}

In stress-primed mice, freezing increased at a greater rate across the session (condition X minutes interaction: F\textsubscript{4,73}\,=\,2.82, p\,=\,0.03; Fig. 5E), while restrained mice exhibited fewer rearing behaviors than non-stressed controls (t\textsubscript{15}\,=\,3.35, p\,=\,0.004; Fig. 5F).

\subsection*{Freezing Bout Duration During TMT Exposure}\label{freezing-bout-duration-during-tmt-exposure}
\addcontentsline{toc}{subsection}{Freezing Bout Duration During TMT Exposure}

\begin{Shaded}
\begin{Highlighting}[]
\NormalTok{a }\OtherTok{\textless{}{-}}\NormalTok{ data }\SpecialCharTok{\%\textgreater{}\%}
  \FunctionTok{na.omit}\NormalTok{() }\SpecialCharTok{\%\textgreater{}\%}
  \FunctionTok{mutate}\NormalTok{(}\AttributeTok{Bins =} \FunctionTok{cut}\NormalTok{(}
\NormalTok{    Start\_clean,}
    \AttributeTok{breaks =} \DecValTok{5}\NormalTok{,}
    \AttributeTok{labels=}\FunctionTok{c}\NormalTok{(}\StringTok{"1"}\NormalTok{,}\StringTok{"2"}\NormalTok{,}\StringTok{"3"}\NormalTok{,}\StringTok{"4"}\NormalTok{,}\StringTok{"5"}\NormalTok{)}
\NormalTok{  )) }\SpecialCharTok{\%\textgreater{}\%}
  \FunctionTok{group\_by}\NormalTok{(ID, Behavior,Condition, Bins) }\SpecialCharTok{\%\textgreater{}\%}
  \FunctionTok{filter}\NormalTok{(Behavior }\SpecialCharTok{!=} \StringTok{"groom"}\NormalTok{) }\SpecialCharTok{\%\textgreater{}\%}
  \FunctionTok{filter}\NormalTok{(Behavior }\SpecialCharTok{!=} \StringTok{"rear"}\NormalTok{)}

\NormalTok{res }\OtherTok{\textless{}{-}} \FunctionTok{aov}\NormalTok{(Duration }\SpecialCharTok{\textasciitilde{}}\NormalTok{ Condition }\SpecialCharTok{*}\NormalTok{ Bins, }\AttributeTok{data =}\NormalTok{ a)}
\FunctionTok{summary}\NormalTok{(res)}
\end{Highlighting}
\end{Shaded}

\begin{verbatim}
##                 Df Sum Sq Mean Sq F value               Pr(>F)    
## Condition        1    207   207.5  20.308           0.00000786 ***
## Bins             4   1380   345.1  33.772 < 0.0000000000000002 ***
## Condition:Bins   4    161    40.2   3.932              0.00367 ** 
## Residuals      631   6447    10.2                                 
## ---
## Signif. codes:  0 '***' 0.001 '**' 0.01 '*' 0.05 '.' 0.1 ' ' 1
\end{verbatim}

The average duration of freezing episode was longer for stress-primed mice (t\textsubscript{563}\,=\,4.22, p\,\textless\,0.001; Fig. 5G) and increases in the duration of freezing episodes were greater across the five-minute session for mice that were restrained before TMT (condition X time interaction: F\textsubscript{4,631}\,=\,3.93, p\,=\,0.004; Fig. 5H).

\section*{Footshock Priming}\label{footshock-priming}
\addcontentsline{toc}{section}{Footshock Priming}

\subsection*{Time Spent Freezing \& Rearing During TMT}\label{time-spent-freezing-rearing-during-tmt}
\addcontentsline{toc}{subsection}{Time Spent Freezing \& Rearing During TMT}

\begin{Shaded}
\begin{Highlighting}[]
\NormalTok{a }\OtherTok{\textless{}{-}}\NormalTok{ data  }\SpecialCharTok{\%\textgreater{}\%}
  \FunctionTok{filter}\NormalTok{(Behavior }\SpecialCharTok{!=} \StringTok{"groom"}\NormalTok{) }\SpecialCharTok{\%\textgreater{}\%}
  \FunctionTok{filter}\NormalTok{(Behavior }\SpecialCharTok{!=} \StringTok{"start"}\NormalTok{) }\SpecialCharTok{\%\textgreater{}\%}
  \FunctionTok{filter}\NormalTok{(Behavior }\SpecialCharTok{!=} \StringTok{"stop"}\NormalTok{) }\SpecialCharTok{\%\textgreater{}\%}
  \FunctionTok{group\_by}\NormalTok{(ID,Condition,Behavior) }\SpecialCharTok{\%\textgreater{}\%}
  \FunctionTok{summarise}\NormalTok{(}
    \AttributeTok{sum=}\FunctionTok{sum}\NormalTok{(Duration),}
    \AttributeTok{Number=}\FunctionTok{n}\NormalTok{(),}
\NormalTok{  ) }\SpecialCharTok{\%\textgreater{}\%}
  \FunctionTok{mutate}\NormalTok{(}\AttributeTok{Perc =}\NormalTok{ (sum }\SpecialCharTok{/} \DecValTok{300}\NormalTok{)}\SpecialCharTok{*}\DecValTok{100}\NormalTok{) }\SpecialCharTok{\%\textgreater{}\%}
  \FunctionTok{mutate}\NormalTok{(}\AttributeTok{Av\_DUR =}\NormalTok{ (sum }\SpecialCharTok{/}\NormalTok{ Number)) }

\NormalTok{res }\OtherTok{\textless{}{-}} \FunctionTok{aov}\NormalTok{(Perc }\SpecialCharTok{\textasciitilde{}}\NormalTok{ Condition }\SpecialCharTok{*}\NormalTok{ Behavior, }\AttributeTok{data =}\NormalTok{ a)}
\FunctionTok{summary}\NormalTok{(res)}
\end{Highlighting}
\end{Shaded}

\begin{verbatim}
##                    Df Sum Sq Mean Sq F value          Pr(>F)    
## Condition           1     86      86   3.211         0.08528 .  
## Behavior            1   3510    3510 130.696 0.0000000000202 ***
## Condition:Behavior  1    492     492  18.333         0.00024 ***
## Residuals          25    671      27                            
## ---
## Signif. codes:  0 '***' 0.001 '**' 0.01 '*' 0.05 '.' 0.1 ' ' 1
\end{verbatim}

Mice that experienced footshock 10\,days before TMT exposure (Fig. 5I) exhibited increased freezing and reduced rearing in response to TMT (condition X behavior interaction F\textsubscript{1,25}\,=\,18.33, p\,≤\,0.001; Fig. 5J,K,L).

\subsection*{Freezing and Rearing Frequency}\label{freezing-and-rearing-frequency}
\addcontentsline{toc}{subsection}{Freezing and Rearing Frequency}

\begin{Shaded}
\begin{Highlighting}[]
\NormalTok{a }\OtherTok{\textless{}{-}}\NormalTok{ data }\SpecialCharTok{\%\textgreater{}\%}
  \FunctionTok{na.omit}\NormalTok{() }\SpecialCharTok{\%\textgreater{}\%}
  \FunctionTok{mutate}\NormalTok{(}\AttributeTok{Bins =} \FunctionTok{cut}\NormalTok{(}
\NormalTok{    Start\_clean,}
    \AttributeTok{breaks =} \DecValTok{5}\NormalTok{,}
    \AttributeTok{labels=}\FunctionTok{c}\NormalTok{(}\StringTok{"1"}\NormalTok{,}\StringTok{"2"}\NormalTok{,}\StringTok{"3"}\NormalTok{,}\StringTok{"4"}\NormalTok{,}\StringTok{"5"}\NormalTok{)}
\NormalTok{  )) }\SpecialCharTok{\%\textgreater{}\%}
  \FunctionTok{group\_by}\NormalTok{(ID, Behavior,Condition, Bins) }\SpecialCharTok{\%\textgreater{}\%}
  \FunctionTok{summarise}\NormalTok{(}
    \AttributeTok{sum =} \FunctionTok{sum}\NormalTok{ (Duration)}
\NormalTok{  ) }\SpecialCharTok{\%\textgreater{}\%}
  \FunctionTok{mutate}\NormalTok{(}\AttributeTok{Perc =}\NormalTok{ (sum }\SpecialCharTok{/} \DecValTok{60}\NormalTok{)}\SpecialCharTok{*}\DecValTok{100}\NormalTok{ ) }\SpecialCharTok{\%\textgreater{}\%}
  \FunctionTok{filter}\NormalTok{(Behavior }\SpecialCharTok{!=} \StringTok{"groom"}\NormalTok{) }\SpecialCharTok{\%\textgreater{}\%}
  \FunctionTok{filter}\NormalTok{(Behavior }\SpecialCharTok{!=} \StringTok{"rear"}\NormalTok{)}

\NormalTok{res }\OtherTok{\textless{}{-}} \FunctionTok{aov}\NormalTok{(Perc }\SpecialCharTok{\textasciitilde{}}\NormalTok{ Condition }\SpecialCharTok{*}\NormalTok{ Bins, }\AttributeTok{data =}\NormalTok{ a)}
\FunctionTok{summary}\NormalTok{(res)}
\end{Highlighting}
\end{Shaded}

\begin{verbatim}
##                Df Sum Sq Mean Sq F value         Pr(>F)    
## Condition       1   1842  1842.2  11.442        0.00123 ** 
## Bins            4  12552  3138.1  19.491 0.000000000157 ***
## Condition:Bins  4    363    90.9   0.564        0.68936    
## Residuals      64  10304   161.0                           
## ---
## Signif. codes:  0 '***' 0.001 '**' 0.01 '*' 0.05 '.' 0.1 ' ' 1
\end{verbatim}

Shock-primed mice froze more than control mice across the five-minute exposure to TMT (main effect of condition: F\textsubscript{1,64}\,=\,11.44, p\,=\,0.001; Fig. 5M).

\begin{Shaded}
\begin{Highlighting}[]
\NormalTok{a }\OtherTok{\textless{}{-}}\NormalTok{ data  }\SpecialCharTok{\%\textgreater{}\%}
  \FunctionTok{filter}\NormalTok{(Behavior }\SpecialCharTok{!=} \StringTok{"groom"}\NormalTok{) }\SpecialCharTok{\%\textgreater{}\%}
  \FunctionTok{filter}\NormalTok{(Behavior }\SpecialCharTok{!=} \StringTok{"start"}\NormalTok{) }\SpecialCharTok{\%\textgreater{}\%}
  \FunctionTok{filter}\NormalTok{(Behavior }\SpecialCharTok{!=} \StringTok{"stop"}\NormalTok{) }\SpecialCharTok{\%\textgreater{}\%}
  \FunctionTok{group\_by}\NormalTok{(ID,Condition,Behavior) }\SpecialCharTok{\%\textgreater{}\%}
  \FunctionTok{summarise}\NormalTok{(}
    \AttributeTok{sum=}\FunctionTok{sum}\NormalTok{(Duration),}
    \AttributeTok{Number=}\FunctionTok{n}\NormalTok{(),}
\NormalTok{  ) }\SpecialCharTok{\%\textgreater{}\%}
  \FunctionTok{mutate}\NormalTok{(}\AttributeTok{Perc =}\NormalTok{ (sum }\SpecialCharTok{/} \DecValTok{300}\NormalTok{)}\SpecialCharTok{*}\DecValTok{100}\NormalTok{) }\SpecialCharTok{\%\textgreater{}\%}
  \FunctionTok{mutate}\NormalTok{(}\AttributeTok{Av\_DUR =}\NormalTok{ (sum }\SpecialCharTok{/}\NormalTok{ Number)) }

\NormalTok{res }\OtherTok{\textless{}{-}} \FunctionTok{aov}\NormalTok{(Number }\SpecialCharTok{\textasciitilde{}}\NormalTok{ Condition }\SpecialCharTok{*}\NormalTok{ Behavior, }\AttributeTok{data =}\NormalTok{ a)}
\FunctionTok{summary}\NormalTok{(res)}
\end{Highlighting}
\end{Shaded}

\begin{verbatim}
##                    Df Sum Sq Mean Sq F value           Pr(>F)    
## Condition           1    499     499   10.78          0.00303 ** 
## Behavior            1   6538    6538  141.20 0.00000000000887 ***
## Condition:Behavior  1   1679    1679   36.26 0.00000273354262 ***
## Residuals          25   1158      46                             
## ---
## Signif. codes:  0 '***' 0.001 '**' 0.01 '*' 0.05 '.' 0.1 ' ' 1
\end{verbatim}

\begin{Shaded}
\begin{Highlighting}[]
\NormalTok{a }\SpecialCharTok{\%\textgreater{}\%}
  \FunctionTok{group\_by}\NormalTok{(Behavior) }\SpecialCharTok{\%\textgreater{}\%}
  \FunctionTok{pairwise\_t\_test}\NormalTok{(Number }\SpecialCharTok{\textasciitilde{}}\NormalTok{ Condition)}
\end{Highlighting}
\end{Shaded}

\begin{verbatim}
## # A tibble: 2 x 10
##   Behavior .y.   group1 group2    n1    n2       p p.signif   p.adj p.adj.signif
## * <chr>    <chr> <chr>  <chr>  <int> <int>   <dbl> <chr>      <dbl> <chr>       
## 1 freeze   Numb~ Contr~ Shock      8     7 3.05e-4 ***      3.05e-4 ***         
## 2 rear     Numb~ Contr~ Shock      8     6 1.24e-3 **       1.24e-3 **
\end{verbatim}

Prior shock exposure increased the frequency of freezing while reducing the frequency of rearing (condition X behavior interaction: F\textsubscript{1,25}\,=\,36.26, p\,\textless\,0.001; Fig. 5N).

\subsection*{Freezing Bout Duration}\label{freezing-bout-duration-1}
\addcontentsline{toc}{subsection}{Freezing Bout Duration}

\begin{Shaded}
\begin{Highlighting}[]
\NormalTok{a }\OtherTok{\textless{}{-}}\NormalTok{ data }\SpecialCharTok{\%\textgreater{}\%}
  \FunctionTok{na.omit}\NormalTok{() }\SpecialCharTok{\%\textgreater{}\%}
  \FunctionTok{mutate}\NormalTok{(}\AttributeTok{Bins =} \FunctionTok{cut}\NormalTok{(}
\NormalTok{    Start\_clean,}
    \AttributeTok{breaks =} \DecValTok{5}\NormalTok{,}
    \AttributeTok{labels=}\FunctionTok{c}\NormalTok{(}\StringTok{"1"}\NormalTok{,}\StringTok{"2"}\NormalTok{,}\StringTok{"3"}\NormalTok{,}\StringTok{"4"}\NormalTok{,}\StringTok{"5"}\NormalTok{)}
\NormalTok{  )) }\SpecialCharTok{\%\textgreater{}\%}
  \FunctionTok{group\_by}\NormalTok{(ID, Behavior,Condition, Bins) }\SpecialCharTok{\%\textgreater{}\%}
  \FunctionTok{filter}\NormalTok{(Behavior }\SpecialCharTok{!=} \StringTok{"groom"}\NormalTok{) }\SpecialCharTok{\%\textgreater{}\%}
  \FunctionTok{filter}\NormalTok{(Behavior }\SpecialCharTok{!=} \StringTok{"rear"}\NormalTok{)}

\NormalTok{res }\OtherTok{\textless{}{-}} \FunctionTok{aov}\NormalTok{(Duration }\SpecialCharTok{\textasciitilde{}}\NormalTok{ Condition }\SpecialCharTok{*}\NormalTok{ Bins, }\AttributeTok{data =}\NormalTok{ a)}
\FunctionTok{summary}\NormalTok{(res)}
\end{Highlighting}
\end{Shaded}

\begin{verbatim}
##                 Df Sum Sq Mean Sq F value              Pr(>F)    
## Condition        1   12.7   12.70   3.163              0.0758 .  
## Bins             4  460.0  115.00  28.643 <0.0000000000000002 ***
## Condition:Bins   4   43.0   10.74   2.675              0.0311 *  
## Residuals      608 2441.2    4.02                                
## ---
## Signif. codes:  0 '***' 0.001 '**' 0.01 '*' 0.05 '.' 0.1 ' ' 1
\end{verbatim}

Additionally, shock-primed mice exhibited shorter rearing episodes than controls (t\textsubscript{124}\,=\,6.36, p\,\textless\,0.001; Fig. 5O), prior shock exposure altered the progression of freezing bout durations across the session (condition X time interaction: F\textsubscript{4,608}\,=\,2.67, p\,=\,0.03; Fig. 5P).

  \bibliography{book.bib,packages.bib}

\end{document}
